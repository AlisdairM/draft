%!TEX root = std.tex
\normannex{depr}{Compatibility features}

\rSec1[depr.general]{General}

\pnum
This Annex describes features of this document
that are specified for compatibility with existing implementations.

\pnum
These are deprecated features, where
\term{deprecated}
is defined as:
Normative for the current revision of \Cpp{},
but having been identified as a candidate for removal from future revisions.
An implementation may declare library names and entities described in this Clause with the
\tcode{deprecated} attribute\iref{dcl.attr.deprecated}.

\rSec1[depr.local]{Non-local use of TU-local entities}

\pnum
A declaration of a non-TU-local entity that is an exposure\iref{basic.link}
is deprecated.
\begin{note}
Such a declaration in an importable module unit is ill-formed.
\end{note}
\begin{example}
\begin{codeblock}
namespace {
  struct A {
    void f() {}
  };
}
A h();                          // deprecated: not internal linkage
inline void g() {A().f();}      // deprecated: inline and not internal linkage
\end{codeblock}
\end{example}

\rSec1[depr.capture.this]{Implicit capture of \tcode{*this} by reference}

\pnum
For compatibility with prior revisions of \Cpp{},
a \grammarterm{lambda-expression} with \grammarterm{capture-default}
\tcode{=}\iref{expr.prim.lambda.capture} may implicitly capture
\tcode{*this} by reference.
\begin{example}
\begin{codeblock}
struct X {
  int x;
  void foo(int n) {
    auto f = [=]() { x = n; };          // deprecated: \tcode{x} means \tcode{this->x}, not a copy thereof
    auto g = [=, this]() { x = n; };    // recommended replacement
  }
};
\end{codeblock}
\end{example}

\rSec1[depr.volatile.type]{Deprecated \tcode{volatile} types}

\pnum
Postfix \tcode{++} and \tcode{--} expressions\iref{expr.post.incr} and
prefix \tcode{++} and \tcode{--} expressions\iref{expr.pre.incr}
of volatile-qualified arithmetic and pointer types are deprecated.

\begin{example}
\begin{codeblock}
volatile int velociraptor;
++velociraptor;                     // deprecated
\end{codeblock}
\end{example}


\pnum
Certain assignments
where the left operand is a volatile-qualified non-class type
are deprecated; see~\ref{expr.assign}.

\begin{example}
\begin{codeblock}
int neck, tail;
volatile int brachiosaur;
brachiosaur = neck;                 // OK
tail = brachiosaur;                 // OK
tail = brachiosaur = neck;          // deprecated
brachiosaur += neck;                // OK
\end{codeblock}
\end{example}


\pnum
A function type\iref{dcl.fct}
with a parameter with volatile-qualified type or
with a volatile-qualified return type is deprecated.

\begin{example}
\begin{codeblock}
volatile struct amber jurassic();                               // deprecated
void trex(volatile short left_arm, volatile short right_arm);   // deprecated
void fly(volatile struct pterosaur* pteranodon);                // OK
\end{codeblock}
\end{example}


\pnum
A structured binding\iref{dcl.struct.bind} of a volatile-qualified type
is deprecated.

\begin{example}
\begin{codeblock}
struct linhenykus { short forelimb; };
void park(linhenykus alvarezsauroid) {
  volatile auto [what_is_this] = alvarezsauroid;                // deprecated
  // ...
}
\end{codeblock}
\end{example}

\rSec1[depr.ellipsis.comma]{Non-comma-separated ellipsis parameters}

A \grammarterm{parameter-declaration-clause}
of the form
\grammarterm{parameter-declaration-list} \tcode{...}
is deprecated\iref{dcl.fct}.
\begin{example}
\begin{codeblock}
void f(int...);         // deprecated
void g(auto...);        // OK, declares a function parameter pack
void h(auto......);     // deprecated
\end{codeblock}
\end{example}

\rSec1[depr.impldec]{Implicit declaration of copy functions}

\pnum
The implicit definition of a copy constructor\iref{class.copy.ctor}
as defaulted is deprecated if the class has
a user-declared copy assignment operator or
a user-declared destructor\iref{class.dtor}.
The implicit definition of a copy assignment operator\iref{class.copy.assign}
as defaulted is deprecated if the class has
a user-declared copy constructor or
a user-declared destructor.
It is possible that future versions of \Cpp{} will specify
that these implicit definitions are deleted\iref{dcl.fct.def.delete}.

\rSec1[depr.static.constexpr]{Redeclaration of \tcode{static constexpr} data members}

\pnum
For compatibility with prior revisions of \Cpp{}, a \keyword{constexpr}
static data member may be redundantly redeclared outside the class with no
initializer\iref{basic.def,class.static.data}.
This usage is deprecated.
\begin{example}
\begin{codeblock}
struct A {
  static constexpr int n = 5;   // definition (declaration in \CppXIV{})
};

constexpr int A::n;             // redundant declaration (definition in \CppXIV{})
\end{codeblock}
\end{example}

\rSec1[depr.lit]{Literal operator function declarations using an identifier}

\pnum
A \grammarterm{literal-operator-id}\iref{over.literal} of the form
\begin{codeblock}
operator @\grammarterm{unevaluated-string}@ @\grammarterm{identifier}@
\end{codeblock}
is deprecated.

\rSec1[depr.template.template]{\tcode{template} keyword before qualified names}

\pnum
The use of the keyword \keyword{template}
before the qualified name of a class or alias template
without a template argument list is deprecated\iref{temp.names}.

\rSec1[depr.numeric.limits.has.denorm]{\tcode{has_denorm} members in \tcode{numeric_limits}}

\pnum
The following type is defined
in addition to those specified in \libheaderref{limits}:
\begin{codeblock}
namespace std {
  enum @\libglobal{float_denorm_style}@ {
    @\libmember{denorm_indeterminate}{float_denorm_style}@ = -1,
    @\libmember{denorm_absent}{float_denorm_style}@ = 0,
    @\libmember{denorm_present}{float_denorm_style}@ = 1
  };
}
\end{codeblock}

\pnum
\indextext{denormalized value|see{number, subnormal}}%
\indextext{value!denormalized|see{number, subnormal}}%
\indextext{subnormal number|see{number, subnormal}}%
\indextext{number!subnormal}%
The following members are defined
in addition to those specified in \ref{numeric.limits.general}:
\begin{codeblock}
static constexpr float_denorm_style has_denorm = denorm_absent;
static constexpr bool has_denorm_loss = false;
\end{codeblock}

\pnum
The values of \tcode{has_denorm} and \tcode{has_denorm_loss} of
specializations of \tcode{numeric_limits} are unspecified.

\pnum
The following members of the specialization \tcode{numeric_limits<bool>} are defined
in addition to those specified in \ref{numeric.special}:
\indexlibrarymember{float_denorm_style}{numeric_limits}%
\indexlibrarymember{has_denorm_loss}{numeric_limits}%
\begin{codeblock}
static constexpr float_denorm_style has_denorm = denorm_absent;
static constexpr bool has_denorm_loss = false;
\end{codeblock}

\rSec1[depr.c.macros]{Deprecated C macros}

\pnum
The header \libheaderref{cfloat} has the following macros:
\begin{codeblock}
#define @\libmacro{FLT_HAS_SUBNORM}@ @\seebelow@
#define @\libmacro{DBL_HAS_SUBNORM}@ @\seebelow@
#define @\libmacro{LDBL_HAS_SUBNORM}@ @\seebelow@
#define @\libmacro{DECIMAL_DIG}@ @\seebelow@
\end{codeblock}
The header defines these macros the same as
the C standard library header \libheader{float.h}.

\xrefc{5.2.4.2.2, 7.33.5}

\pnum
In addition to being available via inclusion of the \libheader{cfloat} header,
the macros \tcode{INFINITY} and \tcode{NAN} are
available when \libheaderref{cmath} is included.

\xrefc{7.12}

\pnum
The header \libheaderref{stdbool.h} has the following macro:
\begin{codeblock}
#define @\libxmacro{bool_true_false_are_defined}@ 1
\end{codeblock}

\xrefc{7.19}

\rSec1[depr.cerrno]{Deprecated error numbers}

\pnum
The header \libheaderref{cerrno} has the following additional macros:

\begin{codeblock}
#define @\libmacro{ENODATA}@ @\seebelow@
#define @\libmacro{ENOSR}@ @\seebelow@
#define @\libmacro{ENOSTR}@ @\seebelow@
#define @\libmacro{ETIME}@ @\seebelow@
\end{codeblock}

\pnum
The meaning of these macros is defined by the POSIX standard.

\pnum
The following \tcode{enum errc} enumerators are defined
in addition to those specified in \ref{system.error.syn}:

\begin{codeblock}
@\libmember{no_message_available}{errc}@,               // \tcode{ENODATA}
@\libmember{no_stream_resources}{errc}@,                // \tcode{ENOSR}
@\libmember{not_a_stream}{errc}@,                       // \tcode{ENOSTR}
@\libmember{stream_timeout}{errc}@,                     // \tcode{ETIME}
\end{codeblock}

\pnum
The value of each \tcode{enum errc} enumerator above
is the same as the value of the \libheader{cerrno} macro
shown in the above synopsis.

\rSec1[depr.meta.types]{Deprecated type traits}

\pnum
The header \libheaderrefx{type_traits}{meta.type.synop}
has the following addition:

\begin{codeblock}
namespace std {
  template<class T> struct is_trivial;
  template<class T> constexpr bool is_trivial_v = is_trivial<T>::value;
  template<class T> struct is_pod;
  template<class T> constexpr bool is_pod_v = is_pod<T>::value;
  template<size_t Len, size_t Align = @\exposid{default-alignment}@> // \seebelow
    struct aligned_storage;
  template<size_t Len, size_t Align = @\exposid{default-alignment}@> // \seebelow
    using @\libglobal{aligned_storage_t}@ = typename aligned_storage<Len, Align>::type;
  template<size_t Len, class... Types>
    struct aligned_union;
  template<size_t Len, class... Types>
    using @\libglobal{aligned_union_t}@ = typename aligned_union<Len, Types...>::type;
}
\end{codeblock}

\pnum
The behavior of a program that adds specializations for
any of the templates defined in this subclause is undefined,
unless explicitly permitted by the specification of the corresponding template.

\pnum
\label{term.trivial.type}%
A \defnadj{trivial}{class} is a class that is trivially copyable and
has one or more eligible default constructors, all of which are trivial.
\begin{note}
In particular,
a trivial class does not have virtual functions or virtual base classes.
\end{note}
A \defnadj{trivial}{type} is a scalar type, a trivial class,
an array of such a type, or a cv-qualified version of one of these types.

\pnum
\indextext{POD}%
A \term{POD class} is a class that is both a trivial class and a
standard-layout class, and has no non-static data members of type non-POD class
(or array thereof). A \term{POD type} is a scalar type, a POD class, an array
of such a type, or a cv-qualified version of one of these types.

\indexlibraryglobal{is_trivial}%
\begin{itemdecl}
template<class T> struct is_trivial;
\end{itemdecl}

\begin{itemdescr}
\pnum
\expects
\tcode{remove_all_extents_t<T>} shall be a complete type or \cv{} \keyword{void}.

\pnum
\remarks
\tcode{is_trivial<T>} is a \oldconcept{UnaryTypeTrait}\iref{meta.rqmts}
with a base characteristic of \tcode{true_type}
if \tcode{T} is a trivial type,
and \tcode{false_type} otherwise.

\pnum
\begin{note}
It is unspecified
whether a closure type\iref{expr.prim.lambda.closure} is a trivial type.
\end{note}
\end{itemdescr}

\indexlibraryglobal{is_pod}%
\begin{itemdecl}
template<class T> struct is_pod;
\end{itemdecl}

\begin{itemdescr}
\pnum
\expects
\tcode{remove_all_extents_t<T>} shall be a complete type or \cv{} \keyword{void}.

\pnum
\remarks
\tcode{is_pod<T>} is a \oldconcept{UnaryTypeTrait}\iref{meta.rqmts}
with a base characteristic of \tcode{true_type}
if \tcode{T} is a POD type,
and \tcode{false_type} otherwise.

\pnum
\begin{note}
It is unspecified whether a closure type\iref{expr.prim.lambda.closure} is a POD type.
\end{note}
\end{itemdescr}

\indexlibraryglobal{aligned_storage}%
\begin{itemdecl}
template<size_t Len, size_t Align = @\exposid{default-alignment}@>
  struct aligned_storage;
\end{itemdecl}

\begin{itemdescr}
\pnum
The value of \exposid{default-alignment} is the most
stringent alignment requirement for any object type whose size
is no greater than \tcode{Len}\iref{basic.types}.

\pnum
\mandates
\tcode{Len} is not zero.
\tcode{Align} is equal to \tcode{alignof(T)} for some type \tcode{T} or
to \exposid{default-alignment}.

\pnum
The member typedef \tcode{type} denotes a trivial standard-layout type
suitable for use as uninitialized storage for any object
whose size is at most \tcode{Len} and
whose alignment is a divisor of \tcode{Align}.

\pnum
\begin{note}
Uses of \tcode{aligned_storage<Len, Align>::type} can be replaced
by an array \tcode{std::byte[Len]} declared with \tcode{alignas(Align)}.
\end{note}

\pnum
\begin{note}
A typical implementation would define \tcode{aligned_storage} as:
\begin{codeblock}
template<size_t Len, size_t Alignment>
struct aligned_storage {
  typedef struct {
    alignas(Alignment) unsigned char __data[Len];
  } type;
};
\end{codeblock}
\end{note}

\end{itemdescr}

\indexlibraryglobal{aligned_union}%
\begin{itemdecl}
template<size_t Len, class... Types>
  struct aligned_union;
\end{itemdecl}

\begin{itemdescr}
\pnum
\mandates
At least one type is provided.
Each type in the template parameter pack \tcode{Types}
is a complete object type.

\pnum
The member typedef \tcode{type} denotes a trivial standard-layout type
suitable for use as uninitialized storage for any object
whose type is listed in \tcode{Types};
its size shall be at least \tcode{Len}.
The static member \tcode{alignment_value}
is an integral constant of type \tcode{size_t}
whose value is the strictest alignment of all types listed in \tcode{Types}.
\end{itemdescr}

\rSec1[depr.relops]{Relational operators}%
\indexlibraryglobal{rel_ops}%

\pnum
The header \libheaderref{utility} has the following additions:

\begin{codeblock}
namespace std::rel_ops {
  template<class T> bool operator!=(const T&, const T&);
  template<class T> bool operator> (const T&, const T&);
  template<class T> bool operator<=(const T&, const T&);
  template<class T> bool operator>=(const T&, const T&);
}
\end{codeblock}

\pnum
To avoid redundant definitions of \tcode{operator!=} out of \tcode{operator==}
and operators \tcode{>}, \tcode{<=}, and \tcode{>=} out of \tcode{operator<},
the library provides the following:

\indexlibrary{\idxcode{operator"!=}}%
\begin{itemdecl}
template<class T> bool operator!=(const T& x, const T& y);
\end{itemdecl}

\begin{itemdescr}
\pnum
\expects
\tcode{T} meets the \oldconcept{EqualityComparable} requirements (\tref{cpp17.equalitycomparable}).

\pnum
\returns
\tcode{!(x == y)}.
\end{itemdescr}

\indexlibraryglobal{operator>}%
\begin{itemdecl}
template<class T> bool operator>(const T& x, const T& y);
\end{itemdecl}

\begin{itemdescr}
\pnum
\expects
\tcode{T} meets the \oldconcept{LessThanComparable} requirements (\tref{cpp17.lessthancomparable}).

\pnum
\returns
\tcode{y < x}.
\end{itemdescr}

\indexlibrary{\idxcode{operator<=}}%
\begin{itemdecl}
template<class T> bool operator<=(const T& x, const T& y);
\end{itemdecl}

\begin{itemdescr}
\pnum
\expects
\tcode{T} meets the \oldconcept{LessThanComparable} requirements (\tref{cpp17.lessthancomparable}).

\pnum
\returns
\tcode{!(y < x)}.
\end{itemdescr}

\indexlibrary{\idxcode{operator>=}}%
\begin{itemdecl}
template<class T> bool operator>=(const T& x, const T& y);
\end{itemdecl}

\begin{itemdescr}
\pnum
\expects
\tcode{T} meets the \oldconcept{LessThanComparable} requirements (\tref{cpp17.lessthancomparable}).

\pnum
\returns
\tcode{!(x < y)}.
\end{itemdescr}

\rSec1[depr.tuple]{Tuple}

\pnum
The header \libheaderref{tuple} has the following additions:

\begin{codeblock}
namespace std {
  template<class T> struct tuple_size<volatile T>;
  template<class T> struct tuple_size<const volatile T>;

  template<size_t I, class T> struct tuple_element<I, volatile T>;
  template<size_t I, class T> struct tuple_element<I, const volatile T>;
}
\end{codeblock}

\begin{itemdecl}
template<class T> struct tuple_size<volatile T>;
template<class T> struct tuple_size<const volatile T>;
\end{itemdecl}

\begin{itemdescr}
\pnum
Let \tcode{TS} denote \tcode{tuple_size<T>} of the cv-unqualified type \tcode{T}.
If the expression \tcode{TS::value} is well-formed
when treated as an unevaluated operand\iref{term.unevaluated.operand},
then specializations of each of the two templates meet
the \oldconcept{TransformationTrait} requirements with a base characteristic of
\tcode{integral_constant<size_t, TS::value>}.
Otherwise, they have no member \tcode{value}.

\pnum
Access checking is performed as if
in a context unrelated to \tcode{TS} and \tcode{T}.
Only the validity of the immediate context of the expression is considered.

\pnum
In addition to being available via inclusion of the \libheaderref{tuple} header,
the two templates are available when any of the headers
\libheaderref{array},
\libheaderref{ranges}, or
\libheaderref{utility}
are included.
\end{itemdescr}

\begin{itemdecl}
template<size_t I, class T> struct tuple_element<I, volatile T>;
template<size_t I, class T> struct tuple_element<I, const volatile T>;
\end{itemdecl}

\begin{itemdescr}
\pnum
Let \tcode{TE} denote \tcode{tuple_element_t<I, T>}
of the cv-unqualified type \tcode{T}.
Then specializations of each of the two templates meet
the \oldconcept{TransformationTrait} requirements
with a member typedef \tcode{type} that names the following type:
\begin{itemize}
\item for the first specialization, \tcode{add_volatile_t<TE>}, and
\item for the second specialization, \tcode{add_cv_t<TE>}.
\end{itemize}

\pnum
In addition to being available via inclusion of the \libheaderref{tuple} header,
the two templates are available when any of the headers
\libheaderref{array},
\libheaderref{ranges}, or
\libheaderref{utility}
are included.
\end{itemdescr}

\rSec1[depr.variant]{Variant}

\pnum
The header \libheaderref{variant} has the following additions:

\begin{codeblock}
namespace std {
  template<class T> struct variant_size<volatile T>;
  template<class T> struct variant_size<const volatile T>;

  template<size_t I, class T> struct variant_alternative<I, volatile T>;
  template<size_t I, class T> struct variant_alternative<I, const volatile T>;
}
\end{codeblock}

\begin{itemdecl}
template<class T> struct variant_size<volatile T>;
template<class T> struct variant_size<const volatile T>;
\end{itemdecl}

\begin{itemdescr}
\pnum
Let \tcode{VS} denote \tcode{variant_size<T>}
of the cv-unqualified type \tcode{T}.
Then specializations of each of the two templates meet
the \oldconcept{UnaryTypeTrait} requirements
with a base characteristic of \tcode{integral_constant<size_t, VS::value>}.
\end{itemdescr}

\begin{itemdecl}
template<size_t I, class T> struct variant_alternative<I, volatile T>;
template<size_t I, class T> struct variant_alternative<I, const volatile T>;
\end{itemdecl}

\begin{itemdescr}
\pnum
Let \tcode{VA} denote \tcode{variant_alternative<I, T>}
of the cv-unqualified type \tcode{T}.
Then specializations of each of the two templates meet
the \oldconcept{TransformationTrait} requirements
with a member typedef \tcode{type} that names the following type:
\begin{itemize}
\item for the first specialization, \tcode{add_volatile_t<VA::type>}, and
\item for the second specialization, \tcode{add_cv_t<VA::type>}.
\end{itemize}
\end{itemdescr}

\rSec1[depr.iterator]{Deprecated \tcode{iterator} class template}

\pnum
The header \libheaderrefx{iterator}{iterator.synopsis} has the following addition:

\indexlibraryglobal{iterator}%
\begin{codeblock}
namespace std {
  template<class Category, class T, class Distance = ptrdiff_t,
           class Pointer = T*, class Reference = T&>
  struct iterator {
    using iterator_category = Category;
    using value_type        = T;
    using difference_type   = Distance;
    using pointer           = Pointer;
    using reference         = Reference;
  };
}
\end{codeblock}

\pnum
The
\tcode{iterator}
template may be used as a base class to ease the definition of required types
for new iterators.

\pnum
\begin{note}
If the new iterator type is a class template, then these aliases
will not be visible from within the iterator class's template definition, but
only to callers of that class.
\end{note}

\pnum
\begin{example}
If a \Cpp{} program wants to define a bidirectional iterator for some data
structure containing \tcode{double} and such that it works on a large memory
model of the implementation, it can do so with:

\begin{codeblock}
class MyIterator :
  public iterator<bidirectional_iterator_tag, double, long, T*, T&> {
  // code implementing \tcode{++}, etc.
};
\end{codeblock}
\end{example}

\rSec1[depr.move.iter.elem]{Deprecated \tcode{move_iterator} access}

\pnum
The following member is declared in addition to those members
specified in \ref{move.iter.elem}:

\begin{codeblock}
namespace std {
  template<class Iterator>
  class move_iterator {
  public:
    constexpr pointer operator->() const;
  };
}
\end{codeblock}

\indexlibrarymember{operator->}{move_iterator}%
\begin{itemdecl}
constexpr pointer operator->() const;
\end{itemdecl}

\begin{itemdescr}
\pnum
\returns
\tcode{current}.
\end{itemdescr}

\rSec1[depr.locale.category]{Deprecated locale category facets}

\pnum
The \tcode{ctype} locale category includes the following facets
as if they were specified
in \tref{locale.category.facets} of \ref{locale.category}.

\begin{codeblock}
codecvt<char16_t, char, mbstate_t>
codecvt<char32_t, char, mbstate_t>
codecvt<char16_t, char8_t, mbstate_t>
codecvt<char32_t, char8_t, mbstate_t>
\end{codeblock}

\pnum
The \tcode{ctype} locale category includes the following facets
as if they were specified
in \tref{locale.spec} of \ref{locale.category}.

\begin{codeblock}
codecvt_byname<char16_t, char, mbstate_t>
codecvt_byname<char32_t, char, mbstate_t>
codecvt_byname<char16_t, char8_t, mbstate_t>
codecvt_byname<char32_t, char8_t, mbstate_t>
\end{codeblock}

\pnum
The following class template specializations are required
in addition to those specified in~\ref{locale.codecvt}.
\indextext{UTF-8}%
\indextext{UTF-16}%
The specializations \tcode{codecvt<char16_t, char, mbstate_t>} and
\tcode{codecvt<char16_t, char8_t, mbstate_t>}
convert between the UTF-16 and UTF-8 encoding forms, and
\indextext{UTF-32}%
the specializations \tcode{codecvt<char32_t, char, mbstate_t>} and
\tcode{codecvt<char32_t, char8_t, mbstate_t>}
convert between the UTF-32 and UTF-8 encoding forms.

\rSec1[depr.format]{Deprecated formatting}

\rSec2[depr.format.syn]{Header \tcode{<format>} synopsis}

\pnum
The header \libheaderref{format} has the following additions:

\begin{codeblock}
namespace std {
  template<class Visitor, class Context>
    decltype(auto) visit_format_arg(Visitor&& vis, basic_format_arg<Context> arg);
}
\end{codeblock}

\rSec2[depr.format.arg]{Formatting arguments}

\indexlibraryglobal{visit_format_arg}%
\begin{itemdecl}
template<class Visitor, class Context>
  decltype(auto) visit_format_arg(Visitor&& vis, basic_format_arg<Context> arg);
\end{itemdecl}

\begin{itemdescr}
\pnum
\effects
Equivalent to: \tcode{return visit(std::forward<Visitor>(vis), arg.value);}
\end{itemdescr}

\rSec1[depr.ctime]{Deprecated time formatting}

\pnum
The header \libheaderref{ctime} has the following additions:
\begin{codeblock}
char* asctime(const tm* timeptr);
char* ctime(const time_t* timer);
\end{codeblock}

\pnum
The functions \tcode{asctime} and \tcode{ctime}
are not required to avoid data races\iref{res.on.data.races}.

\xrefc{7.29}

\rSec1[depr.fs.path.factory]{Deprecated filesystem path factory functions}

\pnum
The header \libheaderrefx{filesystem}{fs.filesystem.syn} has the following additions:

\indexlibraryglobal{u8path}%
\begin{itemdecl}
template<class Source>
  path u8path(const Source& source);
template<class InputIterator>
  path u8path(InputIterator first, InputIterator last);
\end{itemdecl}

\begin{itemdescr}
\pnum
\mandates
The value type of \tcode{Source} and \tcode{InputIterator} is
\tcode{char} or \keyword{char8_t}.

\pnum
\expects
The \tcode{source} and \range{first}{last} sequences are UTF-8 encoded.
\tcode{Source} meets the requirements specified in \ref{fs.path.req}.

\pnum
\returns
\begin{itemize}
\item If \tcode{path::value_type} is \tcode{char} and the current native
      narrow encoding\iref{fs.path.type.cvt} is UTF-8,
      return \tcode{path(source)} or \tcode{path(first, last)};
      otherwise,
\item if \tcode{path::value_type} is \keyword{wchar_t} and the
      native wide encoding is UTF-16, or
      if \tcode{path::value_type} is \keyword{char16_t} or \keyword{char32_t},
      convert \tcode{source} or \range{first}{last}
      to a temporary, \tcode{tmp}, of type \tcode{path::string_type} and
      return \tcode{path(tmp)};
      otherwise,
\item convert \tcode{source} or \range{first}{last}
      to a temporary, \tcode{tmp}, of type \tcode{u32string} and
      return \tcode{path(tmp)}.
\end{itemize}

\pnum
\remarks
Argument format conversion\iref{fs.path.fmt.cvt} applies to the
  arguments for these functions. How Unicode encoding conversions are performed is
  unspecified.

\pnum
\begin{example}
A string is to be read from a database that is encoded in UTF-8, and used
    to create a directory using the native encoding for filenames:
\begin{codeblock}
namespace fs = std::filesystem;
std::string utf8_string = read_utf8_data();
fs::create_directory(fs::u8path(utf8_string));
\end{codeblock}

For POSIX-based operating systems with the native narrow encoding set
    to UTF-8, no encoding or type conversion occurs.

For POSIX-based operating systems with the native narrow encoding not
    set to UTF-8, a conversion to UTF-32 occurs, followed by a conversion to the
    current native narrow encoding. Some Unicode characters may have no native character
    set representation.

For Windows-based operating systems a conversion from UTF-8 to
    UTF-16 occurs.
\end{example}
\begin{note}
The example above is representative of
a historical use of \tcode{filesystem::u8path}.
To indicate a UTF-8 encoding,
passing a \tcode{std::u8string} to \tcode{path}'s constructor is preferred
as it is consistent with \tcode{path}'s handling of other encodings.
\end{note}
\end{itemdescr}

\rSec1[depr.fs.path.obs]{Deprecated filesystem path format observers}

\indexlibraryglobal{path}%
\pnum
The following members are declared in addition to those members
specified in \ref{fs.path.member}:

\begin{codeblock}
namespace std::filesystem {
  class path {
  public:
    std::string string() const;
    std::string generic_string() const;
  };
}
\end{codeblock}

\indexlibrarymember{string}{path}%
\begin{itemdecl}
std::string string() const;
\end{itemdecl}

\begin{itemdescr}
\pnum
\returns
\tcode{system_encoded_string()}.
\end{itemdescr}

\indexlibrarymember{generic_string}{path}%
\begin{itemdecl}
std::string generic_string() const;
\end{itemdecl}

\begin{itemdescr}
\pnum
\returns
\tcode{generic_system_encoded_string()}.
\end{itemdescr}

\rSec1[depr.atomics]{Deprecated atomic operations}

\rSec2[depr.atomics.general]{General}

\pnum
The header \libheaderrefx{atomic}{atomics.syn} has the following additions.

\begin{codeblock}
namespace std {
  template<class T>
    void atomic_init(volatile atomic<T>*, typename atomic<T>::value_type) noexcept;
  template<class T>
    void atomic_init(atomic<T>*, typename atomic<T>::value_type) noexcept;
  template<class T>
    constexpr T kill_dependency(T y) noexcept;                                  // freestanding
  inline constexpr memory_order memory_order_consume = memory_order::consume;   // freestanding

  #define @\libmacro{ATOMIC_VAR_INIT}@(value) @\seebelow@
}
\end{codeblock}

\rSec2[depr.atomics.volatile]{Volatile access}

\pnum
If an \tcode{atomic}\iref{atomics.types.generic} specialization has one of the following overloads,
then that overload participates in overload resolution
even if \tcode{atomic<T>::is_always_lock_free} is \tcode{false}:
\begin{codeblock}
void store(T desired, memory_order order = memory_order::seq_cst) volatile noexcept;
T operator=(T desired) volatile noexcept;
T load(memory_order order = memory_order::seq_cst) const volatile noexcept;
operator T() const volatile noexcept;
T exchange(T desired, memory_order order = memory_order::seq_cst) volatile noexcept;
bool compare_exchange_weak(T& expected, T desired,
                           memory_order success, memory_order failure) volatile noexcept;
bool compare_exchange_strong(T& expected, T desired,
                             memory_order success, memory_order failure) volatile noexcept;
bool compare_exchange_weak(T& expected, T desired,
                           memory_order order = memory_order::seq_cst) volatile noexcept;
bool compare_exchange_strong(T& expected, T desired,
                             memory_order order = memory_order::seq_cst) volatile noexcept;
T fetch_@\placeholdernc{key}@(T operand, memory_order order = memory_order::seq_cst) volatile noexcept;
T operator @\placeholdernc{op}@=(T operand) volatile noexcept;
T* fetch_@\placeholdernc{key}@(ptrdiff_t operand, memory_order order = memory_order::seq_cst) volatile noexcept;
\end{codeblock}

\rSec2[depr.atomics.nonmembers]{Non-member functions}

\indexlibraryglobal{atomic_init}%
\begin{itemdecl}
template<class T>
  void atomic_init(volatile atomic<T>* object, typename atomic<T>::value_type desired) noexcept;
template<class T>
  void atomic_init(atomic<T>* object, typename atomic<T>::value_type desired) noexcept;
\end{itemdecl}

\begin{itemdescr}
\pnum
\effects
Equivalent to: \tcode{atomic_store_explicit(object, desired, memory_order::relaxed);}
\end{itemdescr}

\rSec2[depr.atomics.types.operations]{Operations on atomic types}

\indexlibraryglobal{ATOMIC_VAR_INIT}%
\begin{itemdecl}
#define @\libmacro{ATOMIC_VAR_INIT}@(value) @\seebelow@
\end{itemdecl}

\begin{itemdescr}
\pnum
The macro expands to a token sequence suitable for constant initialization of
an atomic variable with static storage duration of a type that
is initialization-compatible with \tcode{value}.
\begin{note}
This operation possibly needs to initialize locks.
\end{note}
Concurrent access to the variable being initialized,
even via an atomic operation,
constitutes a data race.
\begin{example}
\begin{codeblock}
atomic<int> v = ATOMIC_VAR_INIT(5);
\end{codeblock}
\end{example}
\end{itemdescr}

\rSec2[depr.atomics.order]{\tcode{memory_order::consume}}

\indexlibraryglobal{memory_order}%
\indexlibrarymember{consume}{memory_order}%
\pnum
The \tcode{memory_order} enumeration contains an additional enumerator:
\begin{codeblock}
consume = 1
\end{codeblock}
The \tcode{memory_order::consume} enumerator is allowed wherever
\tcode{memory_order::acquire} is allowed, and it has the same meaning.

\begin{itemdecl}
template<class T> constexpr T kill_dependency(T y) noexcept;
\end{itemdecl}

\begin{itemdescr}
\pnum
\returns
\tcode{y}.
\end{itemdescr}
