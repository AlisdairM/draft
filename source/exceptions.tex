%!TEX root = std.tex
\rSec0[except]{Exception handling}%
\indextext{exception handling|(}

\gramSec[gram.except]{Exception handling}

\indextext{exception object|see{exception handling, exception object}}%
\indextext{object!exception|see{exception handling, exception object}}

\rSec1[except.pre]{Preamble}

\pnum
Exception handling provides a way of transferring control and information
from a point in the execution of a thread to an exception handler
associated with a point previously passed by the execution.
A handler will be invoked only by throwing an exception
in code executed in the handler's try block
or in functions called from the handler's try block.

\indextext{\idxcode{try}}%
%
\begin{bnf}
\nontermdef{try-block}\br
    \keyword{try} compound-statement handler-seq
\end{bnf}

\indextext{\idxcode{try}}%
%
\begin{bnf}
\nontermdef{function-try-block}\br
    \keyword{try} \opt{ctor-initializer} compound-statement handler-seq
\end{bnf}

\begin{bnf}
\nontermdef{handler-seq}\br
    handler \opt{handler-seq}
\end{bnf}

\indextext{\idxcode{catch}}%
%
\begin{bnf}
\nontermdef{handler}\br
    \keyword{catch} \terminal{(} exception-declaration \terminal{)} compound-statement
\end{bnf}

\begin{bnf}
\nontermdef{exception-declaration}\br
    \opt{attribute-specifier-seq} type-specifier-seq declarator\br
    \opt{attribute-specifier-seq} type-specifier-seq \opt{abstract-declarator}\br
    \terminal{...}
\end{bnf}

The optional \grammarterm{attribute-specifier-seq} in an \grammarterm{exception-declaration}
appertains to the parameter of the catch clause\iref{except.handle}.

\pnum
\indextext{exception handling!try block}%
\indextext{exception handling!handler}%
\indextext{try block|see{exception handling, try block}}%
\indextext{handler|see{exception handling, handler}}%
A \grammarterm{try-block} is a \grammarterm{statement}\iref{stmt.pre}.
\begin{note}
Within this Clause
``try block'' is taken to mean both \grammarterm{try-block} and
\grammarterm{function-try-block}.
\end{note}

\pnum
\indextext{exception handling!\idxcode{goto}}%
\indextext{exception handling!\idxcode{switch}}%
\indextext{\idxcode{goto}!and try block}%
\indextext{\idxcode{switch}!and try block}%
\indextext{\idxcode{goto}!and handler}%
\indextext{\idxcode{switch}!and handler}%
The \grammarterm{compound-statement} of a try block or of a handler is a
control-flow-limited statement\iref{stmt.label}.
\begin{example}
\begin{codeblock}
void f() {
  goto l1;          // error
  goto l2;          // error
  try {
    goto l1;        // OK
    goto l2;        // error
    l1: ;
  } catch (...) {
    l2: ;
    goto l1;        // error
    goto l2;        // OK
  }
}

\end{codeblock}
\end{example}
\indextext{\idxcode{goto}!and try block}%
\indextext{\idxcode{switch}!and try block}%
\indextext{\idxcode{return}!and try block}%
\indextext{\idxcode{continue}!and try block}%
\indextext{\idxcode{goto}!and handler}%
\indextext{\idxcode{switch}!and handler}%
\indextext{\idxcode{return}!and handler}%
\indextext{\idxcode{continue}!and handler}%
A
\keyword{goto},
\keyword{break},
\keyword{return},
or
\keyword{continue}
statement can be used to transfer control out of
a try block or handler.
When this happens, each variable declared in the try block
will be destroyed in the context that
directly contains its declaration.
\begin{example}
\begin{codeblock}
lab:  try {
  T1 t1;
  try {
    T2 t2;
    if (@\grammarterm{condition}@)
      goto lab;
    } catch(...) { @\tcode{/* handler 2 */}@ }
  } catch(...) { @\tcode{/* handler 1 */}@ }
\end{codeblock}

Here, executing
\tcode{goto lab;}
will destroy first
\tcode{t2},
then
\tcode{t1},
assuming the
\grammarterm{condition}
does not declare a variable.
Any exception thrown while destroying
\tcode{t2}
will result in executing
\tcode{handler 2};
any exception thrown while destroying
\tcode{t1}
will result in executing
\tcode{handler 1}.
\end{example}

\pnum
\indextext{function try block|see{exception handling, function try block}}%
\indextext{exception handling!function try block}%
A
\grammarterm{function-try-block}
associates a
\grammarterm{handler-seq}
with the
\grammarterm{ctor-initializer},
if present, and the
\grammarterm{compound-statement}.
An exception
thrown during the execution of the
\grammarterm{compound-statement}
or, for constructors and destructors, during the initialization or
destruction, respectively, of the class's subobjects,
transfers control to a handler in a
\grammarterm{function-try-block}
in the same way as an exception thrown during the execution of a
\grammarterm{try-block}
transfers control to other handlers.
\begin{example}
\begin{codeblock}
int f(int);
class C {
  int i;
  double d;
public:
  C(int, double);
};

C::C(int ii, double id)
try : i(f(ii)), d(id) {
    // constructor statements
} catch (...) {
    // handles exceptions thrown from the ctor-initializer and from the constructor statements
}
\end{codeblock}
\end{example}

\pnum
In this Clause, ``before'' and ``after'' refer to the
``sequenced before'' relation\iref{intro.execution}.

\rSec1[except.throw]{Throwing an exception}%
\indextext{exception handling!throwing}%
\indextext{throwing|see{exception handling, throwing}}

\pnum
Throwing an exception transfers control to a handler.
\begin{note}
An exception can be thrown from one of the following contexts:
\grammarterm{throw-expression}{s}\iref{expr.throw},
allocation functions\iref{basic.stc.dynamic.allocation},
\keyword{dynamic_cast}\iref{expr.dynamic.cast},
\keyword{typeid}\iref{expr.typeid},
\grammarterm{new-expression}{s}\iref{expr.new}, and standard library
functions\iref{structure.specifications}.
\end{note}
An object is passed and the type of that object determines which handlers
can catch it.
\begin{example}
\begin{codeblock}
throw "Help!";
\end{codeblock}
can be caught by a
\grammarterm{handler}
of
\keyword{const}
\tcode{\keyword{char}*}
type:
\begin{codeblock}
try {
    // ...
} catch(const char* p) {
    // handle character string exceptions here
}
\end{codeblock}
and
\begin{codeblock}
class Overflow {
public:
    Overflow(char,double,double);
};

void f(double x) {
    throw Overflow('+',x,3.45e107);
}
\end{codeblock}
can be caught by a handler for exceptions of type
\tcode{Overflow}:
\begin{codeblock}
try {
    f(1.2);
} catch(Overflow& oo) {
    // handle exceptions of type \tcode{Overflow} here
}
\end{codeblock}
\end{example}

\pnum
\indextext{exception handling!throwing}%
\indextext{exception handling!handler}%
\indextext{exception handling!nearest handler}%
When an exception is thrown, control is transferred to the nearest handler with
a matching type\iref{except.handle}; ``nearest'' means the handler
for which the
\grammarterm{compound-statement} or
\grammarterm{ctor-initializer}
following the
\keyword{try}
keyword was most recently entered by the thread of control and not yet exited.

\pnum
Throwing an exception
initializes an object with dynamic storage duration,
called the
\defnx{exception object}{exception handling!exception object}.
If the type of the exception object would be
an incomplete type\iref{basic.types.general},
an abstract class type\iref{class.abstract},
or a pointer to an incomplete type other than
\cv{}~\keyword{void}\iref{basic.compound},
the program is ill-formed.

\pnum
\indextext{exception handling!memory}%
\indextext{exception handling!rethrowing}%
\indextext{exception handling!exception object}%
The memory for the exception object is
allocated in an unspecified way, except as noted in~\ref{basic.stc.dynamic.allocation}.
If a handler exits by rethrowing, control is passed to another handler for
the same exception object.
The points of potential destruction for the exception object are:
\begin{itemize}
\item
when an active handler for the exception exits by
any means other than
rethrowing,
immediately after the destruction of the object (if any)
declared in the \grammarterm{exception-declaration} in the handler;

\item
when an object of type \tcode{std::exception_ptr}\iref{propagation}
that refers to the exception object is destroyed,
before the destructor of \tcode{std::exception_ptr} returns.
\end{itemize}

Among all points of potential destruction for the exception object,
there is an unspecified last one
where the exception object is destroyed.
All other points happen before that last one\iref{intro.races}.
\begin{note}
No other thread synchronization is implied in exception handling.
\end{note}
The implementation may then
deallocate the memory for the exception object; any such deallocation
is done in an unspecified way.
\begin{note}
A thrown exception does not
propagate to other threads unless caught, stored, and rethrown using
appropriate library functions; see~\ref{propagation} and~\ref{futures}.
\end{note}

\pnum
\indextext{exception handling!exception object!constructor}%
\indextext{exception handling!exception object!destructor}%
Let \tcode{T} denote the type of the exception object.
Copy-initialization of an object of type \tcode{T} from
an lvalue of type \tcode{const T} in a context unrelated to \tcode{T}
shall be well-formed.
If \tcode{T} is a class type,
the selected constructor is odr-used\iref{basic.def.odr} and
the destructor of \tcode{T} is potentially invoked\iref{class.dtor}.

\pnum
\indextext{exception handling!uncaught}%
An exception is considered \defnx{uncaught}{uncaught exception}
after completing the initialization of the exception object
until completing the activation of a handler for the exception\iref{except.handle}.
\begin{note}
As a consequence, an exception is considered uncaught
during any stack unwinding resulting from it being thrown.
\end{note}

\pnum
\indexlibraryglobal{uncaught_exceptions}%
If an exception is rethrown\iref{expr.throw,propagation},
it is considered uncaught from the point of rethrow
until the rethrown exception is caught.
\begin{note}
The function \tcode{std::uncaught_exceptions}\iref{uncaught.exceptions}
returns the number of uncaught exceptions in the current thread.
\end{note}

\pnum
\indextext{exception handling!rethrow}%
\indextext{rethrow|see{exception handling, rethrow}}%
An exception is considered caught when a handler for that exception
becomes active\iref{except.handle}.
\begin{note}
An exception can have active handlers and still be considered uncaught if
it is rethrown.
\end{note}

\pnum
\indextext{exception handling!terminate called@\tcode{terminate} called}%
\indextext{\idxcode{terminate}!called}%
If the exception handling mechanism
handling an uncaught exception
directly invokes a function that exits via an
exception, the function \tcode{std::terminate} is invoked\iref{except.terminate}.
\begin{example}
\begin{codeblock}
struct C {
  C() { }
  C(const C&) {
    if (std::uncaught_exceptions()) {
      throw 0;      // throw during copy to handler's \grammarterm{exception-declaration} object\iref{except.handle}
    }
  }
};

int main() {
  try {
    throw C();      // calls \tcode{std::terminate} if construction of the handler's
                    // \grammarterm{exception-declaration} object is not elided\iref{class.copy.elision}
  } catch(C) { }
}
\end{codeblock}
\end{example}
\begin{note}
If a destructor directly invoked by stack unwinding exits via an exception,
\tcode{std::terminate} is invoked.
\end{note}


\rSec1[except.ctor]{Stack unwinding}%
\indextext{exception handling!constructors and destructors}%
\indextext{constructor!exception handling|see{exception handling, constructors and destructors}}%
\indextext{destructor!exception handling|see{exception handling, constructors and destructors}}

\pnum
\indextext{unwinding!stack}%
As control passes from the point where an exception is thrown
to a handler,
objects are destroyed by a process,
specified in this subclause, called \defn{stack unwinding}.

\pnum
Each object with automatic storage duration is destroyed if it has been
constructed, but not yet destroyed,
since the try block was entered.
If an exception is thrown during the destruction of temporaries or
local variables for a \keyword{return} statement\iref{stmt.return},
the destructor for the returned object (if any) is also invoked.
The objects are destroyed in the reverse order of the completion
of their construction.
\begin{example}
\begin{codeblock}
struct A { };

struct Y { ~Y() noexcept(false) { throw 0; } };

A f() {
  try {
    A a;
    Y y;
    A b;
    return {};      // \#1
  } catch (...) {
  }
  return {};        // \#2
}
\end{codeblock}
At \#1, the returned object of type \tcode{A} is constructed.
Then, the local variable \tcode{b} is destroyed\iref{stmt.jump}.
Next, the local variable \tcode{y} is destroyed,
causing stack unwinding,
resulting in the destruction of the returned object,
followed by the destruction of the local variable \tcode{a}.
Finally, the returned object is constructed again at \#2.
\end{example}

\pnum
If the initialization of an object
other than by delegating constructor
is terminated by an exception,
the destructor is invoked for
each of the object's subobjects
that were known to be initialized by the object's initialization and
whose initialization has completed\iref{dcl.init}.
\begin{note}
If such an object has a reference member
that extends the lifetime of a temporary object,
this ends the lifetime of the reference member,
so the lifetime of the temporary object is effectively not extended.
\end{note}
\indextext{subobject!initialized, known to be}%
A subobject is \defn{known to be initialized}
if it is not an anonymous union member and
its initialization is specified
\begin{itemize}
\item in \ref{class.base.init} for initialization by constructor,
\item in \ref{class.copy.ctor} for initialization by defaulted copy/move constructor,
\item in \ref{class.inhctor.init} for initialization by inherited constructor,
\item in \ref{dcl.init.aggr} for aggregate initialization,
\item in \ref{expr.prim.lambda.capture} for the initialization of
the closure object when evaluating a \grammarterm{lambda-expression},
\item in \ref{dcl.init.general} for
default-initialization, value-initialization, or direct-initialization
of an array.
\end{itemize}
\begin{note}
This includes virtual base class subobjects
if the initialization
is for a complete object, and
can include variant members
that were nominated explicitly by
a \grammarterm{mem-initializer} or \grammarterm{designated-initializer-clause} or
that have a default member initializer.
\end{note}
If the destructor of an object is terminated by an exception,
each destructor invocation
that would be performed after executing the body of the destructor\iref{class.dtor} and
that has not yet begun execution
is performed.
\begin{note}
This includes virtual base class subobjects if
the destructor was invoked for a complete object.
\end{note}
The subobjects are destroyed in the reverse order of the completion of
their construction. Such destruction is sequenced before entering a
handler of the \grammarterm{function-try-block} of the constructor or destructor,
if any.

\pnum
If the \grammarterm{compound-statement}
of the \grammarterm{function-body}
of a delegating constructor
for an object exits via
an exception, the object's destructor is invoked.
Such destruction is sequenced before entering a handler of the
\grammarterm{function-try-block} of a delegating constructor for that object, if any.

\pnum
\begin{note}
If the object was allocated by a \grammarterm{new-expression}\iref{expr.new},
the matching deallocation function\iref{basic.stc.dynamic.deallocation},
if any, is called to free the storage occupied by the object.
\end{note}


\rSec1[except.handle]{Handling an exception}
\indextext{exception handling!handler|(}%

\pnum
The
\grammarterm{exception-declaration}
in a
\grammarterm{handler}
describes the type(s) of exceptions that can cause
that
\grammarterm{handler}
to be entered.
\indextext{exception handling!handler!incomplete type in}%
\indextext{exception handling!handler!rvalue reference in}%
\indextext{exception handling!handler!array in}%
\indextext{exception handling!handler!pointer to function in}%
The
\grammarterm{exception-declaration}
shall not denote an incomplete type, an abstract class type, or an rvalue reference type.
The
\grammarterm{exception-declaration}
shall not denote a pointer or reference to an
incomplete type, other than ``pointer to \cv{}~\keyword{void}''.

\pnum
A handler of type
\indextext{array!handler of type}%
``array of \tcode{T}'' or
\indextext{function!handler of type}%
function type \tcode{T}
is adjusted to be of type
``pointer to \tcode{T}''.

\pnum
\indextext{exception handling!handler!match|(}%
A
\grammarterm{handler}
is a match for
an exception object
of type
\tcode{E}
if
\begin{itemize}
\item%
The \grammarterm{handler} is of type \cv{}~\tcode{T} or
\cv{}~\tcode{T\&} and
\tcode{E} and \tcode{T}
are the same type (ignoring the top-level \grammarterm{cv-qualifier}{s}), or
\item%
the \grammarterm{handler} is of type \cv{}~\tcode{T} or
\cv{}~\tcode{T\&} and
\tcode{T} is an unambiguous public base class of \tcode{E}, or
\item%
the \grammarterm{handler} is of type \cv{}~\tcode{T} or \tcode{const T\&}
where \tcode{T} is a pointer or pointer-to-member type and
\tcode{E} is a pointer or pointer-to-member type
that can be converted to \tcode{T} by one or more of
\begin{itemize}

\item%
a standard pointer conversion\iref{conv.ptr} not involving conversions
to pointers to private or protected or ambiguous classes
\item%
a function pointer conversion\iref{conv.fctptr}
\item%
a qualification conversion\iref{conv.qual}, or

\end{itemize}

\item
the \grammarterm{handler} is of type \cv{}~\tcode{T} or \tcode{const T\&} where \tcode{T} is a pointer or pointer-to-member type and \tcode{E} is \tcode{std::nullptr_t}.

\end{itemize}

\begin{note}
A
\grammarterm{throw-expression}
whose operand is an integer literal with value zero does not match a handler of
pointer or pointer-to-member type.
A handler of reference to array or function type
is never a match for any exception object\iref{expr.throw}.
\end{note}

\begin{example}
\begin{codeblock}
class Matherr { @\commentellip@ virtual void vf(); };
class Overflow: public Matherr { @\commentellip@ };
class Underflow: public Matherr { @\commentellip@ };
class Zerodivide: public Matherr { @\commentellip@ };

void f() {
  try {
    g();
  } catch (Overflow oo) {
    // ...
  } catch (Matherr mm) {
    // ...
  }
}
\end{codeblock}
Here, the
\tcode{Overflow}
handler will catch exceptions of type
\tcode{Overflow}
and the
\tcode{Matherr}
handler will catch exceptions of type
\tcode{Matherr}
and of all types publicly derived from
\tcode{Matherr}
including exceptions of type
\tcode{Underflow}
and
\tcode{Zerodivide}.
\end{example}

\pnum
The handlers for a try block are tried in order of appearance.
\begin{note}
This makes it possible to write handlers that can never be
executed, for example by placing a handler for a final derived class after
a handler for a corresponding unambiguous public base class.
\end{note}

\pnum
A
\tcode{...}
in a handler's
\grammarterm{exception-declaration}
specifies a match for any exception.
If present, a
\tcode{...}
handler shall be the last handler for its try block.

\pnum
If no match is found among the handlers for a try block,
the search for a matching
handler continues in a dynamically surrounding try block
of the same thread.

\pnum
\indextext{exception handling!terminate called@\tcode{terminate} called}%
\indextext{\idxcode{terminate}!called}%
If the search for a handler
exits the function body of a function with a
non-throwing exception specification,
the function \tcode{std::terminate}\iref{except.terminate} is invoked.
\begin{note}
An implementation is not permitted to reject an expression merely because, when
executed, it throws or might
throw an exception from a function with a non-throwing exception specification.
\end{note}
\begin{example}
\begin{codeblock}
extern void f();                // potentially-throwing

void g() noexcept {
  f();                          // valid, even if \tcode{f} throws
  throw 42;                     // valid, effectively a call to \tcode{std::terminate}
}
\end{codeblock}
The call to
\tcode{f}
is well-formed despite the possibility for it to throw an exception.
\end{example}

\pnum
If no matching handler is found,
the function \tcode{std::terminate} is invoked;
whether or not the stack is unwound before this invocation of
\tcode{std::terminate}
is \impldef{stack unwinding before invocation of
\tcode{std::terminate}}\iref{except.terminate}.

\pnum
A handler is considered \defnx{active}{exception handling!handler!active} when
initialization is complete for the parameter (if any) of the catch clause.
\begin{note}
The stack will have been unwound at that point.
\end{note}
Also, an implicit handler is considered active when
the function \tcode{std::terminate}
is entered due to a throw. A handler is no longer considered active when the
catch clause exits.

\pnum
\indextext{currently handled exception|see{exception handling, currently handled exception}}%
The exception with the most recently activated handler that is
still active is called the
\defnx{currently handled exception}{exception handling!currently handled exception}.

\pnum
Referring to any non-static member or base class of an object
in the handler for a
\grammarterm{function-try-block}
of a constructor or destructor for that object results in undefined behavior.

\pnum
Exceptions thrown in destructors of objects with static storage duration or in
constructors of objects associated with non-block variables with static storage duration are not caught by a
\grammarterm{function-try-block}
on
the \tcode{main} function\iref{basic.start.main}.
Exceptions thrown in destructors of objects with thread storage duration or in constructors of objects associated with non-block variables with thread storage duration are not caught by a
\grammarterm{function-try-block}
on the initial function of the thread.

\pnum
If a \keyword{return} statement\iref{stmt.return} appears in a handler of the
\grammarterm{function-try-block}
of a
constructor, the program is ill-formed.

\pnum
The currently handled exception
is rethrown if control reaches the end of a handler of the
\grammarterm{function-try-block}
of a constructor or destructor.
Otherwise, flowing off the end of
the \grammarterm{compound-statement}
of a \grammarterm{handler}
of a \grammarterm{function-try-block}
is equivalent to flowing off the end of
the \grammarterm{compound-statement}
of that function (see \ref{stmt.return}).

\pnum
The variable declared by the \grammarterm{exception-declaration}, of type
\cv{}~\tcode{T} or \cv{}~\tcode{T\&}, is initialized from the exception object,
of type \tcode{E}, as follows:
\begin{itemize}
\item
if \tcode{T} is a base class of \tcode{E},
the variable is copy-initialized\iref{dcl.init}
from an lvalue of type \tcode{T} designating the corresponding base class subobject
of the exception object;
\item otherwise, the variable is copy-initialized\iref{dcl.init}
from an lvalue of type \tcode{E} designating the exception object.
\end{itemize}

The lifetime of the variable ends
when the handler exits, after the
destruction of any objects with automatic storage duration initialized
within the handler.

\pnum
When the handler declares an object,
any changes to that object will not affect the exception object.
When the handler declares a reference to an object,
any changes to the referenced object are changes to the
exception object and will have effect should that object be rethrown.%
\indextext{exception handling!handler!match|)}%
\indextext{exception handling!handler|)}

\rSec1[except.spec]{Exception specifications}%
\indextext{exception specification|(}

\pnum
The predicate indicating whether a function cannot exit via an exception
is called the \defn{exception specification} of the function.
If the predicate is false,
the function has a
\indextext{exception specification!potentially-throwing}%
\defnx{potentially-throwing exception specification}%
{potentially-throwing!exception specification},
otherwise it has a
\indextext{exception specification!non-throwing}%
\defn{non-throwing exception specification}.
The exception specification is either defined implicitly,
or defined explicitly
by using a \grammarterm{noexcept-specifier}
as a suffix of a function declarator\iref{dcl.fct}.

\begin{bnf}
\nontermdef{noexcept-specifier}\br
    \keyword{noexcept} \terminal{(} constant-expression \terminal{)}\br
    \keyword{noexcept}
\end{bnf}

\pnum
\indextext{exception specification!noexcept!constant expression and}%
In a \grammarterm{noexcept-specifier}, the \grammarterm{constant-expression},
if supplied, shall be a contextually converted constant expression
of type \keyword{bool}\iref{expr.const};
that constant expression is the exception specification of
the function type in which the \grammarterm{noexcept-specifier} appears.
A \tcode{(} token that follows \keyword{noexcept} is part of the
\grammarterm{noexcept-specifier} and does not commence an
initializer\iref{dcl.init}.
The \grammarterm{noexcept-specifier} \keyword{noexcept}
without a \grammarterm{constant-expression}
is
equivalent to the \grammarterm{noexcept-specifier}
\tcode{\keyword{noexcept}(\keyword{true})}.
\begin{example}
\begin{codeblock}
void f() noexcept(sizeof(char[2])); // error: narrowing conversion of value 2 to type \keyword{bool}
void g() noexcept(sizeof(char));    // OK, conversion of value 1 to type \keyword{bool} is non-narrowing
\end{codeblock}
\end{example}

\pnum
If a declaration of a function
does not have a \grammarterm{noexcept-specifier},
the declaration has a potentially throwing exception specification
unless it is a destructor or a deallocation function
or is defaulted on its first declaration,
in which cases the exception specification
is as specified below
and no other declaration for that function
shall have a \grammarterm{noexcept-specifier}.
In an explicit instantiation\iref{temp.explicit}
a \grammarterm{noexcept-specifier} may be specified,
but is not required.
If a \grammarterm{noexcept-specifier} is specified
in an explicit instantiation,
the exception specification shall be the same as
the exception specification of all other declarations of that function.
A diagnostic is required only if the
exception specifications are not the same
within a single translation unit.

\pnum
\indextext{exception specification!virtual function and}%
If a virtual function has a
non-throwing exception specification,
all declarations, including the definition, of any function
that overrides that virtual function in any derived class
shall have a non-throwing
exception specification,
unless the overriding function is defined as deleted.
\begin{example}
\begin{codeblock}
struct B {
  virtual void f() noexcept;
  virtual void g();
  virtual void h() noexcept = delete;
};

struct D: B {
  void f();                     // error
  void g() noexcept;            // OK
  void h() = delete;            // OK
};
\end{codeblock}

The declaration of
\tcode{D::f}
is ill-formed because it
has a potentially-throwing exception specification,
whereas
\tcode{B::f}
has a non-throwing exception specification.
\end{example}

\pnum
An expression $E$ is
\defnx{potentially-throwing}{potentially-throwing!expression} if
\begin{itemize}
\item
$E$ is a function call\iref{expr.call}
whose \grammarterm{postfix-expression}
has a function type,
or a pointer-to-function type,
with a potentially-throwing exception specification,
or
\item
$E$ implicitly invokes a function
(such as an overloaded operator,
an allocation function in a \grammarterm{new-expression},
a constructor for a function argument,
or a destructor)
that has a potentially-throwing exception specification,
or
\item
$E$ is a \grammarterm{throw-expression}\iref{expr.throw},
or
\item
$E$ is a \keyword{dynamic_cast} expression that casts to a reference type and
requires a runtime check\iref{expr.dynamic.cast},
or
\item
$E$ is a \keyword{typeid} expression applied to a
(possibly parenthesized) built-in unary \tcode{*} operator
applied to a pointer to a
polymorphic class type\iref{expr.typeid},
or
\item
any of the immediate subexpressions\iref{intro.execution}
of $E$ is potentially-throwing.
\end{itemize}

\pnum
An implicitly-declared constructor for a class \tcode{X},
or a constructor without a \grammarterm{noexcept-specifier}
that is defaulted on its first declaration,
has a potentially-throwing exception specification
if and only if
any of the following constructs is potentially-throwing:
\begin{itemize}
\item
the invocation of a constructor selected by overload resolution
in the implicit definition of the constructor
for class \tcode{X}
to initialize a potentially constructed subobject, or
\item
a subexpression of such an initialization,
such as a default argument expression, or,
\item
for a default constructor, a default member initializer.
\end{itemize}
\begin{note}
Even though destructors for fully-constructed subobjects
are invoked when an exception is thrown
during the execution of a constructor\iref{except.ctor},
their exception specifications do not contribute
to the exception specification of the constructor,
because an exception thrown from such a destructor
would call the function \tcode{std::terminate}
rather than escape the constructor\iref{except.throw,except.terminate}.
\end{note}

\pnum
The exception specification for an implicitly-declared destructor,
or a destructor without a \grammarterm{noexcept-specifier},
is potentially-throwing if and only if
any of the destructors
for any of its potentially constructed subobjects
has a potentially-throwing exception specification or
the destructor is virtual and the destructor of any virtual base class
has a potentially-throwing exception specification.

\pnum
The exception specification for an implicitly-declared assignment operator,
or an assignment-operator without a \grammarterm{noexcept-specifier}
that is defaulted on its first declaration,
is potentially-throwing if and only if
the invocation of any assignment operator
in the implicit definition is potentially-throwing.

\pnum
A deallocation function\iref{basic.stc.dynamic.deallocation}
with no explicit \grammarterm{noexcept-specifier}
has a non-throwing exception specification.

\pnum
The exception specification for a comparison operator function\iref{over.binary}
without a \grammarterm{noexcept-specifier}
that is defaulted on its first declaration
is potentially-throwing if and only if
any expression
in the implicit definition is potentially-throwing.

\pnum
\begin{example}
\begin{codeblock}
struct A {
  A(int = (A(5), 0)) noexcept;
  A(const A&) noexcept;
  A(A&&) noexcept;
  ~A();
};
struct B {
  B() noexcept;
  B(const B&) = default;        // implicit exception specification is \tcode{\keyword{noexcept}(\keyword{true})}
  B(B&&, int = (throw 42, 0)) noexcept;
  ~B() noexcept(false);
};
int n = 7;
struct D : public A, public B {
    int * p = new int[n];
    // \tcode{D::D()} potentially-throwing, as the \keyword{new} operator may throw \tcode{bad_alloc} or \tcode{bad_array_new_length}
    // \tcode{D::D(const D\&)} non-throwing
    // \tcode{D::D(D\&\&)} potentially-throwing, as the default argument for \tcode{B}'s constructor may throw
    // \tcode{D::\~D()} potentially-throwing
};
\end{codeblock}
Furthermore, if
\tcode{A::\~{}A()}
were virtual,
the program would be ill-formed since a function that overrides a virtual
function from a base class
shall not have a potentially-throwing exception specification
if the base class function has a non-throwing exception specification.
\end{example}

\pnum
An exception specification is considered to be \defnx{needed}{needed!exception specification} when:
\begin{itemize}
\item in an expression, the function is selected by
overload resolution\iref{over.match,over.over};

\item the function is odr-used\iref{term.odr.use};

\item the exception specification is compared to that of another
declaration (e.g., an explicit specialization or an overriding virtual
function);

\item the function is defined; or

\item the exception specification is needed for a defaulted
function that calls the function.
\begin{note}
A defaulted declaration does not require the
exception specification of a base member function to be evaluated
until the implicit exception specification of the derived
function is needed, but an explicit \grammarterm{noexcept-specifier} needs
the implicit exception specification to compare against.
\end{note}
\end{itemize}
The exception specification of a defaulted
function is evaluated as described above only when needed; similarly, the
\grammarterm{noexcept-specifier} of a specialization
of a templated function
is instantiated only when needed.
%
\indextext{exception specification|)}

\rSec1[except.special]{Special functions}

\rSec2[except.special.general]{General}

\pnum
The function \tcode{std::terminate}\iref{except.terminate}
is used by the exception
handling mechanism for coping with errors related to the exception handling
mechanism itself.
The function \tcode{std::uncaught_exceptions}\iref{uncaught.exceptions}
reports how many exceptions are uncaught in the current thread.
The function \tcode{std::current_exception}\iref{propagation} and the class
\tcode{std::nested_exception}\iref{except.nested} can be used by a program to
capture the currently handled exception.

\rSec2[except.terminate]{The \tcode{std::terminate} function}

\pnum
\indextext{\idxcode{terminate}}%
Some errors in a program cannot be recovered from, such as when an exception
is not handled or a \tcode{std::thread} object is destroyed while its thread
function is still executing. In such cases,
the function \tcode{std::terminate}\iref{exception.terminate} is invoked.
\begin{note}
These situations are:
\indextext{\idxcode{terminate}!called}%
\begin{itemize}
\item%
when the exception handling mechanism, after completing
the initialization of the exception object
but before
activation of a handler for the exception\iref{except.throw},
calls a function that exits
via an exception, or

\item%
when the exception handling mechanism cannot find a handler for a thrown exception\iref{except.handle}, or

\item when the search for a handler\iref{except.handle}
exits the function body of a function
with a non-throwing exception specification\iref{except.spec},
including when a contract-violation handler
invoked from an evaluation of
a function contract assertion\iref{basic.contract.eval} associated with the function
exits via an exception,
or

\item%
when the destruction of an object during stack unwinding\iref{except.ctor}
terminates by throwing an exception, or

\item%
when initialization of a non-block
variable with static or thread storage duration\iref{basic.start.dynamic}
exits via an exception, or

\item%
when destruction of an object with static or thread storage duration exits
via an exception\iref{basic.start.term}, or

\item%
when execution of a function registered with
\tcode{std::atexit} or \tcode{std::at_quick_exit}
exits via an exception\iref{support.start.term}, or

\item%
when a
\grammarterm{throw-expression}\iref{expr.throw}
with no operand attempts to rethrow an exception and no exception is being
handled\iref{except.throw}, or

\item%
when the function \tcode{std::nested_exception::rethrow_nested} is called for an object
that has captured no exception\iref{except.nested}, or

\item%
when execution of the initial function of a thread exits via
an exception\iref{thread.thread.constr}, or

\item%
for a parallel algorithm whose \tcode{ExecutionPolicy} specifies such
behavior\iref{execpol.seq,execpol.par,execpol.parunseq},
when execution of an element access function\iref{algorithms.parallel.defns}
of the parallel algorithm exits via an exception\iref{algorithms.parallel.exceptions}, or

\item%
when the destructor or the move assignment operator is invoked on an object
of type \tcode{std::thread} that refers to
a joinable thread\iref{thread.thread.destr,thread.thread.assign}, or

\item%
when a call to a \tcode{wait()}, \tcode{wait_until()}, or \tcode{wait_for()}
function on a condition variable\iref{thread.condition.condvar,thread.condition.condvarany}
fails to meet a postcondition, or

\item%
when a callback invocation exits via an exception
when requesting stop on
a \tcode{std::stop_source} or
a \tcode{std::in\-place_stop_source}\iref{stopsource.mem,stopsource.inplace.mem},
or in the constructor of
\tcode{std::stop_callback} or
\tcode{std::inplace_stop_callback}\iref{stopcallback.cons,stopcallback.inplace.cons}
when a callback invocation exits via an exception, or

\item%
when a \tcode{run_loop} object is destroyed
that is still in the \tcode{running} state\iref{exec.run.loop}, or

\item%
when \tcode{unhandled_stopped} is called on
a \tcode{with_awaitable_senders<T>} object\iref{exec.with.awaitable.senders}
whose continuation is not a handle to a coroutine
whose promise type has an \tcode{unhandled_stopped} member function, or

\item%
when an object \tcode{scope} of type
\tcode{std::execution::simple_counting_scope} or
\tcode{std::execution::counting_scope}
is destroyed and
\tcode{scope.\exposid{state}} is not equal to
\exposid{joined},
\exposid{unused}, or
\exposid{unused-and-closed}\iref{exec.simple.counting.ctor}, or

\item%
when \tcode{std::execution::get_parallel_scheduler} is called and
\tcode{std::execution::system_context_replace\-ability::query_parallel_scheduler_backend()}
returns a null pointer value\iref{exec.par.scheduler}, or

\item%
when an exception is thrown from a coroutine \tcode{std::execution::task}\iref{exec.task}
which doesn't support a \tcode{std::execution::set_error_t(std::exception_ptr)} completion.
\end{itemize}
\end{note}

\pnum
\indextext{\idxcode{terminate}}%
In the situation where no matching handler is found, it is
\impldef{stack unwinding before invocation of \tcode{std::terminate}}
whether or not the stack is unwound
before \tcode{std::terminate} is invoked.
In the situation where the search for a handler\iref{except.handle}
exits the function body of a function
with a non-throwing exception specification\iref{except.spec}, it is
\impldef{whether stack is unwound before invoking the function \tcode{std::terminate}
when a \tcode{noexcept} specification is violated}
whether the stack is unwound, unwound partially, or not unwound at all
before the function \tcode{std::terminate} is invoked.
In all other situations, the stack shall not be unwound before
the function \tcode{std::terminate}
is invoked.
An implementation is not permitted to finish stack unwinding
prematurely based on a determination that the unwind process
will eventually cause an invocation of the function
\tcode{std::terminate}.
\indextext{exception handling|)}
